\documentclass[12pt]{article}
\usepackage{graphicx,times,geometry,float,color,pdfpages,amsmath,hyperref}
\geometry{margin=20mm,nohead}
\begin{document}

\textbf{\Large BIOL 3295 Fall 2020 Syllabus}

\section{Instructor Information}\label{instructor-information}

Instructor: Dr.~Amy Hurford\\
Office: Teaching remotely\\
Email: ahurford@mun.ca
I will try to reply to emails within 24 hours (excluding evenings,
weekends and holidays). I am always available during the lecture times.
Please email to request a meeting for a different time.

\section{Course Information}\label{course-information}

TR 12.00-12.50pm\\
F 1-1.50pm\\
WebEx links for lecture times are on the course Brightspace under
Announcements.

Course description:\\
Population and Evolutionary Ecology is an introduction to the theory and
principles of evolutionary ecology and population dynamics.
Pre-requisites: BIOL 2600; at least one of BIOL 2010, 2122 or
2210.\\[2\baselineskip]Course format:\\
The course has been re-designed for online delivery. Specifically, no
exams that require invigilation are part of the grading scheme because
these are challenging to deliver remotely. Pre-recorded lectures limit
my ability to interact with students. Therefore, I have elected to
dedicate all lecture time to interacting with students. For each class
there is a list of questions you are required to answer and hand-in.
Prior to some classes there may be a \emph{Required Reading}, that if
completed will allow you to answer the day's assignment questions. Prior
to the day of class you should complete the \emph{Required Reading}. In
addition, I can most effectively help you if you have read over the
questions ahead of time.

Course expectations:\\
Any students that are disruptive, violating university policies, or
acting in a potentially unsafe way will be warned and asked to
leave.\\[2\baselineskip]Learning goals:\\
I consider your completed assignments to be a portfolio of your
knowledge in population and evolutionary ecology. You will also get some
exposure to coding in \texttt{R}. It takes time to become proficient in
a programming language, but the time you will spend coding in this class
will help you towards becoming more proficient. The course content
emphasizes a deeper understanding of fewer concepts. You have the
opportunity to further explore a topic of interest to you for the final
project.

Required Text and Resources:\\
The course materials are online at
\url{https://ahurford.github.io/BIOL-3295-Fall-2020/}. In addition you
will need a computer to install \texttt{R} and \texttt{RStudio}. This
will be covered on Thursday Sept 17 (see Chapter \ref{Rinstall}). Class
announcements and WebEx links will be provided on the course BrightSpace
and your assignments are to be submitted to BrightSpace.

\section{Method of Evaluation}\label{method-of-evaluation}

\begin{itemize}
\tightlist
\item
  27 assignments (equal weighting) - 50\%
\item
  Midterm (due Fri Nov 6 at 5pm) - 15\%
\item
  Final Project (due Monday Dec 14 at 9am) - 35\%
\end{itemize}

You should aim to complete each assignment before the next class, but
assignments will be accepted, without penalty, up to a week later.

Late assignments, labs, and missed midterms, and final exams will be
accommodated as described by University Regulation 6.7.3 and 6.7.5 (see
\url{https://www.mun.ca/regoff/calendar/sectionNo=REGS-0474} for
Regulations).

\section{Additional Policies}\label{additional-policies}

\subsection{Accommodation of students with
disabilities}\label{accommodation-of-students-with-disabilities}

Memorial University of Newfoundland is committed to supporting inclusive
education based on the principles of equity, accessibility and
collaboration. Accommodations are provided within the scope of the
University Policies for the Accommodations for Students with
Disabilities see \url{www.mun.ca/policy/site/policy.php?id=239}.
Students who may need an academic accommodation are asked to initiate
the request with the Glenn Roy Blundon Centre at the earliest
opportunity (see \url{www.mun.ca/blundon} for more information).

\subsection{Academic misconduct}\label{academic-misconduct}

Students are expected to adhere to those principles, which constitute
proper academic conduct. A student has the responsibility to know which
actions, as described under Academic Offences in the University
Regulations, could be construed as dishonest or improper. Students found
guilty of an academic offence may be subject to a number of penalties
commensurate with the offence including reprimand, reduction of grade,
probation, suspension or expulsion from the University. For more
information regarding this policy, students should refer to University
Regulation 6.12.

\subsection{Equity and Diversity}\label{equity-and-diversity}

A safe learning environment will be provided for all students regardless
of race, colour, nationality, ethnic origin, social origin, religious
creed, religion, age, disability, disfigurement, sex (including
pregnancy), sexual orientation, gender identity, gender expression,
marital status, family status, source of income or political opinion.

You should not photograph or record myself, teaching assistants, or
other students in the class without first obtaining permission.
Accommodation will be made for students with special needs.

The sound should be turned off on phones and computers during class.

\section{Additional Supports}\label{additional-supports}

Resources for additional support can be found at:

\begin{itemize}
\item
  \url{www.mun.ca/currentstudents/student/}
\item
  \url{https://munsu.ca/resource-centres/}
\end{itemize}

\section{Tentative course schedule}\label{tentative-course-schedule}

The course schedule is found in the toolbar of the class materials, see
\url{https://ahurford.github.io/BIOL-3295-Fall-2020/}.

The last day to drop the course without academic prejudice is Wednesday
Nov. 4.

\section{Handing in your work}\label{handing-in-your-work}

\subsection{Making figures to hand-in}\label{figures}

The graphs you hand in need to have descriptive axeses and a figure
caption. You may put these elements together using a word processing
software such as \emph{Microsoft Word}.

\subsection{Writing R scripts to hand-in}\label{RScript}

To write your own R scripts follow the guidelines described in Chapter 7
\href{https://ahurford.github.io/quant-guide-all-courses/style.html}{Best
Practices} of \emph{Quantitative training in Biology}. If you are asked
to hand in your R script this means you need to submit an \texttt{.R}
file on Brightspace.

\end{document}
