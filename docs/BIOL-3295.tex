\documentclass[]{book}
\usepackage{lmodern}
\usepackage{amssymb,amsmath}
\usepackage{ifxetex,ifluatex}
\usepackage{fixltx2e} % provides \textsubscript
\ifnum 0\ifxetex 1\fi\ifluatex 1\fi=0 % if pdftex
  \usepackage[T1]{fontenc}
  \usepackage[utf8]{inputenc}
\else % if luatex or xelatex
  \ifxetex
    \usepackage{mathspec}
  \else
    \usepackage{fontspec}
  \fi
  \defaultfontfeatures{Ligatures=TeX,Scale=MatchLowercase}
\fi
% use upquote if available, for straight quotes in verbatim environments
\IfFileExists{upquote.sty}{\usepackage{upquote}}{}
% use microtype if available
\IfFileExists{microtype.sty}{%
\usepackage{microtype}
\UseMicrotypeSet[protrusion]{basicmath} % disable protrusion for tt fonts
}{}
\usepackage[unicode=true]{hyperref}
\hypersetup{
            pdfborder={0 0 0},
            breaklinks=true}
\urlstyle{same}  % don't use monospace font for urls
\usepackage{color}
\usepackage{fancyvrb}
\newcommand{\VerbBar}{|}
\newcommand{\VERB}{\Verb[commandchars=\\\{\}]}
\DefineVerbatimEnvironment{Highlighting}{Verbatim}{commandchars=\\\{\}}
% Add ',fontsize=\small' for more characters per line
\usepackage{framed}
\definecolor{shadecolor}{RGB}{248,248,248}
\newenvironment{Shaded}{\begin{snugshade}}{\end{snugshade}}
\newcommand{\KeywordTok}[1]{\textcolor[rgb]{0.13,0.29,0.53}{\textbf{{#1}}}}
\newcommand{\DataTypeTok}[1]{\textcolor[rgb]{0.13,0.29,0.53}{{#1}}}
\newcommand{\DecValTok}[1]{\textcolor[rgb]{0.00,0.00,0.81}{{#1}}}
\newcommand{\BaseNTok}[1]{\textcolor[rgb]{0.00,0.00,0.81}{{#1}}}
\newcommand{\FloatTok}[1]{\textcolor[rgb]{0.00,0.00,0.81}{{#1}}}
\newcommand{\ConstantTok}[1]{\textcolor[rgb]{0.00,0.00,0.00}{{#1}}}
\newcommand{\CharTok}[1]{\textcolor[rgb]{0.31,0.60,0.02}{{#1}}}
\newcommand{\SpecialCharTok}[1]{\textcolor[rgb]{0.00,0.00,0.00}{{#1}}}
\newcommand{\StringTok}[1]{\textcolor[rgb]{0.31,0.60,0.02}{{#1}}}
\newcommand{\VerbatimStringTok}[1]{\textcolor[rgb]{0.31,0.60,0.02}{{#1}}}
\newcommand{\SpecialStringTok}[1]{\textcolor[rgb]{0.31,0.60,0.02}{{#1}}}
\newcommand{\ImportTok}[1]{{#1}}
\newcommand{\CommentTok}[1]{\textcolor[rgb]{0.56,0.35,0.01}{\textit{{#1}}}}
\newcommand{\DocumentationTok}[1]{\textcolor[rgb]{0.56,0.35,0.01}{\textbf{\textit{{#1}}}}}
\newcommand{\AnnotationTok}[1]{\textcolor[rgb]{0.56,0.35,0.01}{\textbf{\textit{{#1}}}}}
\newcommand{\CommentVarTok}[1]{\textcolor[rgb]{0.56,0.35,0.01}{\textbf{\textit{{#1}}}}}
\newcommand{\OtherTok}[1]{\textcolor[rgb]{0.56,0.35,0.01}{{#1}}}
\newcommand{\FunctionTok}[1]{\textcolor[rgb]{0.00,0.00,0.00}{{#1}}}
\newcommand{\VariableTok}[1]{\textcolor[rgb]{0.00,0.00,0.00}{{#1}}}
\newcommand{\ControlFlowTok}[1]{\textcolor[rgb]{0.13,0.29,0.53}{\textbf{{#1}}}}
\newcommand{\OperatorTok}[1]{\textcolor[rgb]{0.81,0.36,0.00}{\textbf{{#1}}}}
\newcommand{\BuiltInTok}[1]{{#1}}
\newcommand{\ExtensionTok}[1]{{#1}}
\newcommand{\PreprocessorTok}[1]{\textcolor[rgb]{0.56,0.35,0.01}{\textit{{#1}}}}
\newcommand{\AttributeTok}[1]{\textcolor[rgb]{0.77,0.63,0.00}{{#1}}}
\newcommand{\RegionMarkerTok}[1]{{#1}}
\newcommand{\InformationTok}[1]{\textcolor[rgb]{0.56,0.35,0.01}{\textbf{\textit{{#1}}}}}
\newcommand{\WarningTok}[1]{\textcolor[rgb]{0.56,0.35,0.01}{\textbf{\textit{{#1}}}}}
\newcommand{\AlertTok}[1]{\textcolor[rgb]{0.94,0.16,0.16}{{#1}}}
\newcommand{\ErrorTok}[1]{\textcolor[rgb]{0.64,0.00,0.00}{\textbf{{#1}}}}
\newcommand{\NormalTok}[1]{{#1}}
\usepackage{longtable,booktabs}
\usepackage{graphicx,grffile}
\makeatletter
\def\maxwidth{\ifdim\Gin@nat@width>\linewidth\linewidth\else\Gin@nat@width\fi}
\def\maxheight{\ifdim\Gin@nat@height>\textheight\textheight\else\Gin@nat@height\fi}
\makeatother
% Scale images if necessary, so that they will not overflow the page
% margins by default, and it is still possible to overwrite the defaults
% using explicit options in \includegraphics[width, height, ...]{}
\setkeys{Gin}{width=\maxwidth,height=\maxheight,keepaspectratio}
\IfFileExists{parskip.sty}{%
\usepackage{parskip}
}{% else
\setlength{\parindent}{0pt}
\setlength{\parskip}{6pt plus 2pt minus 1pt}
}
\setlength{\emergencystretch}{3em}  % prevent overfull lines
\providecommand{\tightlist}{%
  \setlength{\itemsep}{0pt}\setlength{\parskip}{0pt}}
\setcounter{secnumdepth}{5}
% Redefines (sub)paragraphs to behave more like sections
\ifx\paragraph\undefined\else
\let\oldparagraph\paragraph
\renewcommand{\paragraph}[1]{\oldparagraph{#1}\mbox{}}
\fi
\ifx\subparagraph\undefined\else
\let\oldsubparagraph\subparagraph
\renewcommand{\subparagraph}[1]{\oldsubparagraph{#1}\mbox{}}
\fi

\author{}
\date{\vspace{-2.5em}}

\begin{document}

{
\setcounter{tocdepth}{1}
\tableofcontents
}
\chapter{Thurs Sept 10: Syllabus}\label{thurs-sept-10-syllabus}

\section{Instructor Information}\label{instructor-information}

Instructor: Dr.~Amy Hurford\\
Office: Teaching remotely\\
Email: \href{mailto:ahurford@mun.ca}{\nolinkurl{ahurford@mun.ca}}\\
I will try to reply to emails within 24 hours (excluding weekends and
holidays). Outside of work hours, I am frequently unable to reply. I am
always available during the lecture times. Please email to request a
meeting for a different time.

\section{Course Information}\label{course-information}

TR 12.00-12.50pm\\
F 1-1.50pm\\
WebEx links appear in the course materials on each relevant day\\
Course description: Population and Evolutionary Ecology is an
introduction to the theory and principles of evolutionary ecology and
population dynamics. Pre-requisites: BIOL 2600; at least one of BIOL
2010, 2122 or 2210.\\[2\baselineskip]Course format:\\
The course has been re-designed for online delivery. Specifically, no
exams that require invigilation are part of the grading scheme because
these are challenging to invigilate remotely. Pre-recorded lectures
limit my ability to interact with students. Therefore, I have elected to
dedicate all lecture time to answering your questions. For each day of
`lecture' there is a list of questions you are required to answer and
hand-in, ideally before the next class, but you have up to a week to
submit your answers. There is also list of readings, that if completed,
will allow you to answer the questions. Prior to `lectures' you should
complete the required readings, and I can most effectively help you if
you have read over the questions ahead of time.\\
Course expectations: Any students that are disruptive, violating
university policies, or acting in a potentially unsafe way will be
warned and asked to leave.\\[2\baselineskip]\#\# Learning goals I
consider your completed assignments to be a portfolio of your knowledge
in population and evolutionary ecology. You will also get some exposure
to coding in \texttt{R}. It takes a while to become proficient in a
programming language, but all the time you spend coding helps you
towards becoming proficient. Some students may enjoy this aspect of the
course and subsequently continue to build their programming abilities.
The content emphasizes a deeper understanding of fewer concepts, and
building your ability to learn more on your own.

\section{Required Text and Resources}\label{required-text-and-resources}

The course materials are all (here){[}{]}.

Late assignments, labs, and missed midterms, and final exams will be
accommodated as described by University Regulation 6.7.3 and 6.7.5 (see
\url{https://www.mun.ca/regoff/calendar/sectionNo=REGS-0474} for
Regulations). The Final exam will cover all Lecture material and
readings, but not Labs. Specific regulations governing final
examinations are described by University Regulation 6.8. Lecture
participation is highly recommended and practice problems completed
during lectures may appear on Assignments.

Additional Policies Accommodation of students with disabilities Memorial
University of Newfoundland is committed to supporting inclusive
education based on the principles of equity, accessibility and
collaboration. Accommodations are provided within the scope of the
University Policies for the Accommodations for Students with
Disabilities (www.mun.ca/policy/site/policy.php?id=239). Students who
may need an academic accommodation are asked to initiate the request
with the Glenn Roy Blundon Centre at the earliest opportunity
(www.mun.ca/blundon).

Academic misconduct Students are expected to adhere to those principles,
which constitute proper academic conduct. A student has the
responsibility to know which actions, as described under Academic
Offences in the University Regulations, could be construed as dishonest
or improper. Students found guilty of an academic offence may be subject
to a number of penalties commensurate with the offence including
reprimand, reduction of grade, probation, suspension or expulsion from
the University. For more information regarding this policy, students
should refer to University Regulation 6.12.

Equity and Diversity A safe learning environment will be provided for
all students regardless of race, colour, nationality, ethnic origin,
social origin, religious creed, religion, age, disability,
disfigurement, sex (including pregnancy), sexual orientation, gender
identity, gender expression, marital status, family status, source of
income or political opinion.

You should not photograph or record myself, teaching assistants, or
other students in the class without first obtaining permission.
Accommodation will be made for students with special needs.

The sound should be turned off on phones and computers during class.

Additional Supports Resources for additional support can be found at: •
www.mun.ca/currentstudents/student/ •
\url{https://munsu.ca/resource-centres/}

The last day to drop the course without academic prejudice is Wednesday
Nov. 4.

\section{Grading}\label{grading}

\begin{itemize}
\tightlist
\item
  27 assignments - 50\%
\item
  Midterm - 15\%
\item
  Final Project - 35\%
\end{itemize}

\section{Handing in your work}\label{handing-in-your-work}

\subsection{Making figures to hand-in}\label{figures}

The graphs you hand in need to have descriptive axeses and a figure
caption. You may put these elements together using a word processing
software such as \emph{Microsoft Word}.

\subsection{Writing R scripts to hand-in}\label{RScript}

To write your own R scripts follow the guidelines described in Chapter 7
\href{https://ahurford.github.io/quant-guide-all-courses/style.html}{Best
Practices} of \emph{Quantitative training in Biology}. If you are asked
to hand in your R script this means you need to submit an \texttt{.R}
file to your dropbox on brightspace.

\chapter{Friday Sept 11: Sept What is a
population?}\label{friday-sept-11-sept-what-is-a-population}

Submit your answers to the Questions to Brightspace.

\section{To hand-in}\label{to-hand-in}

\begin{enumerate}
\def\labelenumi{\arabic{enumi}.}
\item
  Give a definition of a population from a textbook or peer-reviewed
  publication. Provide the citation.
\item
  Write 1 paragraph describing why the definition of a population
  matters?
\item
  Find a peer-reviewed paper where a population is studied. Write 1
  paragraph discussing how a population is defined in the study.
\end{enumerate}

\section{Resources}\label{resources}

Vandermeer, J.H., Goldberg, D.E., 2013. Population Ecology: First
Principles (Second Edition). Princeton University Press, Princeton,
United States.
\href{https://ebookcentral-proquest-com.qe2a-proxy.mun.ca/lib/mun/detail.action?docID=1205619}{Link}

The Princeton Guide to Ecology, edited by Simon A. Levin, et al.,
Princeton University Press, 2009. ProQuest Ebook Central,
\href{https://ebookcentral-proquest-com.qe2a-proxy.mun.ca/lib/mun/detail.action?docID=557123}{Link}

Gunn, A., Russell, D. and Eamer, J. 2011. Northern caribou population
trends in Canada. Canadian Biodiversity: Ecosystem Status and Trends
2010, Technical Thematic Report No. 10. Canadian Councils of Resource
Ministers. Ottawa, ON. iv + 71 p.
\href{https://biodivcanada.chm-cbd.net/ecosystem-status-trends-2010/technical-report-10}{Link}

Sacchi, R., Gentilli, A., Razzetti, E., Barbieri, F., 2002. Effects of
building features on density and flock distribution of feral pigeons
Columba livia var. domestica in an urban environment. Can. J. Zool. 80,
48-54. \href{https://doi.org/10.1139/z01-202}{Link}

\chapter{Tues Sept 15: Exponential growth - discrete
time}\label{tues-sept-15-exponential-growth---discrete-time}

\section{Required reading}\label{required-reading}

Vandermeer, J.H., Goldberg, D.E., 2013. Population Ecology: First
Principles (Second Edition). Princeton University Press, Princeton,
United States. p1-3.
\href{https://ebookcentral-proquest-com.qe2a-proxy.mun.ca/lib/mun/detail.action?docID=1205619}{Link}

\section{Questions}\label{questions}

Submit your answers to Brightspace

\begin{enumerate}
\def\labelenumi{\arabic{enumi}.}
\item
  Let \(\lambda = 5\) in equation (3) (see the required reading).
  Explain the meaning of \(\lambda = 5\).
\item
  Suppose the number of lilypads during week 7 is 150. Let
  \(\lambda = 5\), and assume that the units of \(t\) are weeks. Use
  equation (3) to calculate the number of lilypads in week 8.
\item
  Use your answer to question 2. to calculate the number of lilypads in
  week 9.
\item
  Equation (4) for the required reading assumes that \(N_0=1\), however,
  this formula can be generalized such that
\end{enumerate}

\[
N_t = N_0\lambda^t
\]

where \(N_t\) is the population size at time, \(t\). Define time such
that \(t=0\) is week 7 and \(t\) is then the number of weeks since week
7. Use the the equation above to answer question 3 and confirm that the
answer is the same (i.e., find the population size for week 9, when
\(\lambda=5\) and the population size for week 7 is 150).

\begin{enumerate}
\def\labelenumi{\arabic{enumi}.}
\setcounter{enumi}{4}
\item
  Use the formula from question 4 to find the population size for week
  15, where \(N_0\) and \(\lambda\) are the same as question 4.
\item
  It is important to note that all mathematical formulas should have the
  same units on both sides of the equals sign and for each term that is
  added or subtracted. The units of the population size, \(N_t\), at
  time, \(t\), are number. The geometric growth rate, \(\lambda\) is
  unitless. Choose a formula from the required reading and give the
  units for each of the terms to show that both sides of the equals have
  the same units. For example, for the equation that appears in question
  4, we have:
\end{enumerate}

\begin{eqnarray*}
N_t & = & N_0 \lambda^t \\
\left(\mbox{number}\right)& =&(\mbox{number}) (\mbox{unitless})^{weeks}\\
\left(\mbox{number}\right)& =& \left(\mbox{number}\right)\\
\end{eqnarray*}

Note that:

\begin{eqnarray*}
(\mbox{unitless}) \times (\mbox{quantity with units}) & = & (\mbox{quantity with units}) \\ (\mbox{unitless})^{(\mbox{quantity})}& =& (\mbox{unitless})
\end{eqnarray*}

\begin{enumerate}
\def\labelenumi{\arabic{enumi}.}
\setcounter{enumi}{6}
\item
  Although not stated in the reading, \(\lambda = 1 + b - d\) where
  \(b\) is the per capita birth rate over one time step (i.e.~one week
  for this example), and \(d\) is the fraction of the lilypad population
  that dies over one time step. The number 1 is considered unitless,
  what must the units of \(b\) and \(d\) be?
\item
  In the reading, \(\lambda = 2\). Given that \(\lambda = 1 + b - d\),
  what are some possible values of \(b\) and \(d\).
\item
  {[}True or False{]} For discrete time exponential growth (as per the
  reading), the change in population size from one week to the next
  depends not so much on the per capita birth rate, but on the
  difference between the per capita birth rate and the per capita death
  rate.
\end{enumerate}

\chapter{Thurs Sept 17: Getting started with
R}\label{thurs-sept-17-getting-started-with-r}

\section{Required reading}\label{required-reading-1}

You are required to read and complete all the exercises in Chapters 1
\href{https://ahurford.github.io/quant-guide-all-courses/}{\emph{Introduction}},
3
\href{https://ahurford.github.io/quant-guide-all-courses/install.html}{\emph{R
and RStudio}}, and 4
\href{https://ahurford.github.io/quant-guide-all-courses/rstudio.html}{\emph{Finding
your way around RStudio}} of:

\emph{Quantitative skills for biology}
\url{https://ahurford.github.io/quant-guide-all-courses/}

When you are finished you should have \texttt{R} and \texttt{RStudio}
installed on your computer, or you should be familar with running
\texttt{RStudio\ Cloud}.

\section{Questions}\label{questions-1}

\begin{enumerate}
\def\labelenumi{\arabic{enumi}.}
\tightlist
\item
  Write 1 paragraph describing your experience completing the the
  exercises.
\end{enumerate}

\chapter{Friday Sept 18: Protection Island
1}\label{friday-sept-18-protection-island-1}

The information below is taken from the following source: Newcomb, HR.
1940.
\href{https://ir.library.oregonstate.edu/concern/graduate_thesis_or_dissertations/js956j801?locale=en}{Ring-necked
pheasant studies on Protection Island in the Strait of Juan de Fuca},
Washington. MS thesis. Oregon State University.

\begin{enumerate}
\def\labelenumi{\alph{enumi}.}
\tightlist
\item
  Pheasant chicks are born during the summer.
\item
  In May 1937, 10 pheasants were introduced to the island. Before the
  next breeding season there were 35 pheasants.
\item
  November 10, 1938 a census estimated 110 pheasants.
\item
  October 13, 1939 a census estimated 400 pheasants.
\end{enumerate}

\section{Questions}\label{questions-2}

\begin{enumerate}
\def\labelenumi{\arabic{enumi}.}
\tightlist
\item
  Read and complete all the exercises in Chapters 6.3
  \href{https://ahurford.github.io/quant-guide-all-courses/rintro.html\#variables-and-assignment}{\emph{Variables
  and assignment}} to 6.10
  \href{https://ahurford.github.io/quant-guide-all-courses/rintro.html\#r-packages}{\emph{R
  packages}} and 11
  \href{https://ahurford.github.io/quant-guide-all-courses/graph.html}{\emph{Making
  graphs in R}} of \emph{Quantitative skills for biology}
\end{enumerate}

 HAND IN

\begin{itemize}
\tightlist
\item
   Answer all questions marked HAND IN in the reading 
\end{itemize}

\begin{enumerate}
\def\labelenumi{\arabic{enumi}.}
\setcounter{enumi}{1}
\tightlist
\item
  To make a graph of the data listed in b.-d., we need to learn how to
  work with dates. We will consider two possible approaches:
\end{enumerate}

\begin{enumerate}
\def\labelenumi{\roman{enumi}.}
\item
  Use a built-in \texttt{R} function to convert dates to a format that
  can be plotted (this question); and
\item
  Convert the dates to number of days since a reference date. Now the
  dates are numbers and these values can be plotted on the x-axis of a
  graph (question 3).
\end{enumerate}

In this question, we will proceed with option i. The function we will
use is \texttt{as.Date()}. You can learn how to use this function using
an internet search or by typing the following into your
\texttt{Console}:

\begin{Shaded}
\begin{Highlighting}[]
\NormalTok{?as.Date}
\end{Highlighting}
\end{Shaded}

These files can be difficult to understand (see
\href{https://ahurford.github.io/quant-guide-all-courses/help.html\#how-to-interpret-r-help-files}{R
Help files}. A good way to proceed is to experiment with the function in
the \texttt{Console}. Try these:

\begin{Shaded}
\begin{Highlighting}[]
\KeywordTok{as.Date}\NormalTok{(}\DecValTok{2012-01-31}\NormalTok{, }\DataTypeTok{format =} \NormalTok{%Y-%m-%d)}
\KeywordTok{as.Date}\NormalTok{(}\StringTok{"2012-01-31"}\NormalTok{, }\DataTypeTok{format =} \StringTok{"%Y-%m-%d"}\NormalTok{)}
\end{Highlighting}
\end{Shaded}

Note that only the second command is error-free. The first command fails
because the date argument for the \texttt{as.Date()} function must be a
character string, i.e., must be enclosed in \texttt{""} (see
\texttt{?character}).

It is also possible to omit the format argument and just code:
\texttt{as.Date("2012-01-31")}. The help file notes that when the format
argument is not specified, that formats will be tried one by one and an
error will be returned if none work. It is advisable to specify the
format, as allowing the function to infer the format could introduce
errors.

Chapter 6.9
\href{https://ahurford.github.io/quant-guide-all-courses/rintro.html\#data-structures}{Data
structures} describes how to make a vector (note a vector is a list of
numbers rather than just a single number). We need to make a vector of
the dates so that we can make our plot. For example,

\begin{Shaded}
\begin{Highlighting}[]
\NormalTok{x <-}\StringTok{ }\KeywordTok{as.Date}\NormalTok{(}\KeywordTok{c}\NormalTok{(}\StringTok{"2012-01-31"}\NormalTok{, }\StringTok{"2012-03-05"}\NormalTok{, }\StringTok{"2013-01-11"}\NormalTok{), }\DataTypeTok{format =} \StringTok{"%Y-%m-%d"}\NormalTok{)}
\end{Highlighting}
\end{Shaded}

Having completed Chapter 11
\href{https://ahurford.github.io/quant-guide-all-courses/graph.html}{Making
graphs in R}, and having learned how to work with dates, you should now
be able to write an R script to make plot using the information in b.-d.
above.

 HAND IN

\begin{itemize}
\item
   A graph and figure caption, which has dates on the x-axis and the
  pheasant population size on the y-axis drawing from the information
  provided in b.-d. You will need to guess the date of `before the
  breeding season' as stated in b. and you should disclose the value of
  this guess in the figure caption. See \ref{figures} for more
  information. 
\item
   An R Script that produces the figure described above. See
  \ref{RScript} for more information. 
\end{itemize}

\begin{enumerate}
\def\labelenumi{\arabic{enumi}.}
\setcounter{enumi}{2}
\tightlist
\item
  Next, we try approach ii. to make the graph. Under approach ii. we
  will work with the dates by converting them to the number of days
  since a reference date. To do this we will use the \texttt{julian()}
  function, which is part of the \texttt{chron} package.
\end{enumerate}

Read Section 4.4 of
\href{https://ahurford.github.io/quant-guide-all-courses/}{Quantitative
skills for biology} regarding installing packages. Install the package
\texttt{chron} using either the \texttt{Install} button on the
\texttt{Packages} tab, or by using the command
\texttt{install.packages("chron")} in the \texttt{Console} window. Note
that the package is only available for use once you check the box on the
\texttt{Packages} tab or by running the following command in the
\texttt{Console}:

\begin{Shaded}
\begin{Highlighting}[]
\KeywordTok{require}\NormalTok{(}\StringTok{"chron"}\NormalTok{)}
\end{Highlighting}
\end{Shaded}

After the \texttt{chron} package is loaded, we can then query the
\texttt{julian} function,

\begin{Shaded}
\begin{Highlighting}[]
\NormalTok{?julian}
\end{Highlighting}
\end{Shaded}

or use an internet search to better understand how to use it. As the
help files can be difficult to understand, another approach to is to try
out the function. Try the following:

\begin{Shaded}
\begin{Highlighting}[]
\KeywordTok{julian}\NormalTok{(}\DecValTok{1}\NormalTok{,}\DecValTok{1}\NormalTok{,}\DecValTok{1970}\NormalTok{)}
\KeywordTok{julian}\NormalTok{(}\DecValTok{1}\NormalTok{,}\DecValTok{2}\NormalTok{,}\DecValTok{1970}\NormalTok{)}
\KeywordTok{julian}\NormalTok{(}\DecValTok{2}\NormalTok{,}\DecValTok{1}\NormalTok{,}\DecValTok{1970}\NormalTok{)}
\KeywordTok{julian}\NormalTok{(}\DecValTok{1}\NormalTok{,}\DecValTok{1}\NormalTok{,}\DecValTok{1971}\NormalTok{)}
\KeywordTok{julian}\NormalTok{(}\DecValTok{1}\NormalTok{,}\DecValTok{1}\NormalTok{,}\DecValTok{1969}\NormalTok{)}
\end{Highlighting}
\end{Shaded}

Which argument position of \texttt{julian()} function corresponds to the
month? Note also that by default the origin (the origin is the value of
0) is set to January 1, 1970. Experiment by running the following lines
of code:

\begin{Shaded}
\begin{Highlighting}[]
\KeywordTok{julian}\NormalTok{(}\DecValTok{1}\NormalTok{,}\DecValTok{1}\NormalTok{,}\DecValTok{2000}\NormalTok{)}
\KeywordTok{julian}\NormalTok{(}\DecValTok{1}\NormalTok{,}\DecValTok{1}\NormalTok{,}\DecValTok{2000}\NormalTok{, }\DataTypeTok{origin =} \KeywordTok{c}\NormalTok{(}\DecValTok{1}\NormalTok{,}\DecValTok{1}\NormalTok{,}\DecValTok{1970}\NormalTok{))}
\KeywordTok{julian}\NormalTok{(}\DecValTok{1}\NormalTok{,}\DecValTok{1}\NormalTok{,}\DecValTok{2000}\NormalTok{, }\DataTypeTok{origin =} \KeywordTok{c}\NormalTok{(}\DecValTok{1}\NormalTok{,}\DecValTok{1}\NormalTok{,}\DecValTok{2000}\NormalTok{))}
\end{Highlighting}
\end{Shaded}

Finally, we need to make our figure. Recall that the plot function
requires vectors of equal length for the x- and y-axes. Make a vector of
the days since a reference as follows:

\begin{Shaded}
\begin{Highlighting}[]
\NormalTok{ref.day =}\StringTok{ }\KeywordTok{c}\NormalTok{(}\DecValTok{1}\NormalTok{,}\DecValTok{1}\NormalTok{,}\DecValTok{2000}\NormalTok{)}
\NormalTok{x =}\StringTok{ }\KeywordTok{c}\NormalTok{(}\KeywordTok{julian}\NormalTok{(}\DecValTok{1}\NormalTok{,}\DecValTok{1}\NormalTok{,}\DecValTok{2000}\NormalTok{, }\DataTypeTok{origin =} \NormalTok{ref.day), }\KeywordTok{julian}\NormalTok{(}\DecValTok{1}\NormalTok{,}\DecValTok{1}\NormalTok{,}\DecValTok{2002}\NormalTok{, }\DataTypeTok{origin =} \NormalTok{ref.day))}
\end{Highlighting}
\end{Shaded}

If you run into problems you can query the value of \texttt{x} in your
console, and you can use \texttt{length(x)} to check the length of
\texttt{x}.

 HAND IN

\begin{itemize}
\item
   As for question 2. you need to hand in a graph with descriptive axes
  and with a figure caption. The y-axis on your graph is population size
  and the x-axis will be created using the \texttt{julian()} function.
  Be sure to label the x-axis differently than you did in question 2. 
\item
   You also need to produce an R Script that makes the figure described
  above. See \ref{RScript} for more information. Name this file
  \emph{protection-island-q3.R} 
\end{itemize}

\begin{enumerate}
\def\labelenumi{\arabic{enumi}.}
\setcounter{enumi}{3}
\tightlist
\item
  For the discrete time exponential (geometric) growth model, ideally
  the census points should be a fixed amount of time apart, i.e., a day,
  a week, or a year.
\end{enumerate}

 HAND IN

\begin{itemize}
\tightlist
\item
  Provide 1-3 sentences discussing the timing of the census points for
  Protection Island. 
\end{itemize}

\chapter{Tues Sept 22: Protection Island
2}\label{tues-sept-22-protection-island-2}

Here is some additional information also taken from: Newcomb, HR. 1940.
\href{https://ir.library.oregonstate.edu/concern/graduate_thesis_or_dissertations/js956j801?locale=en}{Ring-necked
pheasant studies on Protection Island in the Strait of Juan de Fuca},
Washington. MS thesis. Oregon State University.

\begin{enumerate}
\def\labelenumi{\alph{enumi}.}
\tightlist
\item
  Pheasant chicks are born during the summer.
\item
  In May 1937, 10 pheasants were introduced to the island. Before the
  next breeding season there were 35.
\item
  November 10, 1938 a census estimated 110 pheasants.
\item
  October 13, 1939 a census estimated 400 pheasants.
\item
  Between the 1938 and 1939 censuses, Newcomb observed that 17 adult
  birds died.
\item
  During the 1938 nesting season: 5.86 eggs/nest. 83.57\% of eggs
  hatched.
\item
  During the 1939 nesting season: 8.73 eggs/nest. 64.58\% hatched.
\item
  During the 1939 nesting season: Average number of chicks per clutch
  was 6.93.\(^1\)
\item
  You can assume the sex ratio is 50:50 male to female. Pheasants are a
  sexually reproducing species.
\end{enumerate}

\(^1\) Note that g. and h. appear to be contradictory.

\section{Questions}\label{questions-3}

\begin{enumerate}
\def\labelenumi{\arabic{enumi}.}
\item
  Let \(d\) be the fraction of population that dies each year. What is
  \(d\) for the ring-tailed pheasant population on Protection Island?
  Write down any assumptions you have made.
\item
  \(b\) is the per capita number of births each year. What is the value
  of \(b\)? Write down any assumptions you have made.
\item
  Recall that \(\lambda = 1 + b-d\). What is the value of \(\lambda\)?
  Is this population is exected to grow over time?
\item
  Lets assume that the pheasant population on Protection Island grows
  geoemetrically (i.e.~exponentially) where the geometric growth rate,
  \(\lambda\), is the value that you estimated in 5. Lets predict the
  population size each May beginning with May 1937. Let \(N_0 = 10\) and
  let \(t\) be the number of years since May 1937. Recall that when a
  population grows geometrically,
\end{enumerate}

\[N_{t} = N_0 \lambda^t \]

You can use \texttt{R} to do this calculation as follows (you should use
your value of \(\lambda\) from question 5):

\begin{Shaded}
\begin{Highlighting}[]
\NormalTok{t <-}\DecValTok{1}
\NormalTok{N0 <-}\DecValTok{10}
\NormalTok{lambda <-}\DecValTok{3}
\NormalTok{N0*lambda^t}
\end{Highlighting}
\end{Shaded}

where since \(t = 1\) the result of \texttt{N0*lambda\^{}t} is
\(N_{t+1}\), with \(t=1\), such that \texttt{N0*lambda\^{}t} is the
value of \(N_{t+1}=N_2\): the population size two years after May 1937.
You can change the value of \(t\) and repeat the calculation. Unless you
have cleared your workshop it won't be necessary to re-input
\(N_0 = 10\) and \(\lambda = 3\). As such, you can calculate \(N_3\)
with the following commands:

\begin{Shaded}
\begin{Highlighting}[]
\NormalTok{t <-}\DecValTok{2}
\NormalTok{N0*lambda^t}
\end{Highlighting}
\end{Shaded}

 HAND IN

\begin{itemize}
\tightlist
\item
  Use \texttt{R} to predict the value of the pheasant population size
  every year up until May 1940.
\end{itemize}

\begin{enumerate}
\def\labelenumi{\arabic{enumi}.}
\setcounter{enumi}{6}
\tightlist
\item
  The approach to calculating the pheasant population size in Question 6
  is not very organized. In this question, we will learn how to make a
  data frame, use a for loop, and use the function \texttt{rbind()}.
\end{enumerate}

Read
\href{https://ahurford.github.io/quantitative-training-guide/rintro.html\#data-structures}{Data
structures} in \emph{Quantitative training for biology}.

Create a one row dataframe called \texttt{df}:

\begin{Shaded}
\begin{Highlighting}[]
\NormalTok{df <-}\StringTok{ }\KeywordTok{data.frame}\NormalTok{(}\DataTypeTok{time =} \DecValTok{0}\NormalTok{, }\DataTypeTok{popn.size =} \DecValTok{10}\NormalTok{)}
\end{Highlighting}
\end{Shaded}

Query \texttt{df} in your \texttt{Console} to see the data frame you
have created. We would like to add successive values of the population
size that we calculate to the data frame. To do this we use the
\texttt{rbind()} function, which binds rows together.

\begin{Shaded}
\begin{Highlighting}[]
\NormalTok{new.result <-}\StringTok{ }\KeywordTok{data.frame}\NormalTok{(}\DataTypeTok{time =} \DecValTok{1}\NormalTok{, }\DataTypeTok{popn.size =} \DecValTok{20}\NormalTok{)}
\NormalTok{df <-}\StringTok{ }\KeywordTok{rbind}\NormalTok{(df, new.result)}
\end{Highlighting}
\end{Shaded}

Here the \texttt{rbind()} function takes the \texttt{df} dataframe and
adds the \texttt{new.result} data frame as a new row onto the bottom.
Note that the code above \emph{overwrites} the value of \texttt{df}:
that is, \texttt{new.result} is added to the bottom of the \texttt{df}
dataframe (containing only one row), and the result is called
\texttt{df} (which now has two rows), and the old dataframe \texttt{df}
(with one row) is overwritten. As such, each time you run the command
\texttt{df\ \textless{}-\ rbind(df,\ new.result)} another row is added
to \texttt{df}. Try the following:

\begin{Shaded}
\begin{Highlighting}[]
\NormalTok{new.result <-}\StringTok{ }\KeywordTok{data.frame}\NormalTok{(}\DataTypeTok{time =} \DecValTok{1}\NormalTok{, }\DataTypeTok{popn.size =} \DecValTok{20}\NormalTok{)}
\NormalTok{df <-}\StringTok{ }\KeywordTok{rbind}\NormalTok{(df, new.result)}
\NormalTok{df <-}\StringTok{ }\KeywordTok{rbind}\NormalTok{(df, new.result)}
\NormalTok{df <-}\StringTok{ }\KeywordTok{rbind}\NormalTok{(df, new.result)}
\end{Highlighting}
\end{Shaded}

If you query the value of \texttt{df} you can see that the several rows,
all with identical values have been added because we have run the
command \texttt{df\ \textless{}-\ rbind(df,\ new.result)} multiple times
while the value of \texttt{new.result} is unchanged. Now let's change
the value of \texttt{new.result} between each time we run the
\texttt{df\ \textless{}-\ rbind(df,\ new.result)} command.

\begin{Shaded}
\begin{Highlighting}[]
\NormalTok{new.result <-}\StringTok{ }\KeywordTok{data.frame}\NormalTok{(}\DataTypeTok{time =} \DecValTok{1}\NormalTok{, }\DataTypeTok{popn.size =} \DecValTok{20}\NormalTok{)}
\NormalTok{df <-}\StringTok{ }\KeywordTok{rbind}\NormalTok{(df, new.result)}
\NormalTok{new.result <-}\StringTok{ }\KeywordTok{data.frame}\NormalTok{(}\DataTypeTok{time =} \DecValTok{2}\NormalTok{, }\DataTypeTok{popn.size =} \DecValTok{30}\NormalTok{)}
\NormalTok{df <-}\StringTok{ }\KeywordTok{rbind}\NormalTok{(df, new.result)}
\end{Highlighting}
\end{Shaded}

Finally, when we do calculations for a sequence of values, it is easier
to code this using a \texttt{for} loop.

\begin{Shaded}
\begin{Highlighting}[]
\NormalTok{lambda <-}\StringTok{ }\FloatTok{1.2}
\NormalTok{N0 <-}\StringTok{ }\DecValTok{10}
\NormalTok{df <-}\StringTok{ }\KeywordTok{data.frame}\NormalTok{(}\DataTypeTok{time =} \DecValTok{0}\NormalTok{, }\DataTypeTok{popn.size =} \DecValTok{10}\NormalTok{)}
\NormalTok{for(t in }\KeywordTok{seq}\NormalTok{(}\DecValTok{1}\NormalTok{,}\DecValTok{4}\NormalTok{,}\DecValTok{1}\NormalTok{))\{}
  \NormalTok{val <-}\StringTok{ }\NormalTok{N0*lambda^t}
  \NormalTok{new.result <-}\StringTok{ }\KeywordTok{data.frame}\NormalTok{(}\DataTypeTok{time =} \NormalTok{t, }\DataTypeTok{popn.size =} \NormalTok{val)}
  \NormalTok{df <-}\StringTok{ }\KeywordTok{rbind}\NormalTok{(df, new.result)}
\NormalTok{\}}
\end{Highlighting}
\end{Shaded}

To understand the above code, after copy and pasting it into your
\texttt{Console}, query the value of \texttt{df}: you should see
predicted population sizes up until 4 years after May 1937. Now, lets
try to understand \texttt{seq(1,4,1)}. Let's learn about the
\texttt{seq()} function by trying it out in the \texttt{Console}. What
is the result of each of these?

\begin{Shaded}
\begin{Highlighting}[]
\KeywordTok{seq}\NormalTok{(-}\DecValTok{10}\NormalTok{,}\DecValTok{10}\NormalTok{)}
\KeywordTok{seq}\NormalTok{(-}\DecValTok{10}\NormalTok{,}\DecValTok{5}\NormalTok{,}\FloatTok{0.1}\NormalTok{)}
\end{Highlighting}
\end{Shaded}

The \texttt{for} loop works by beginning with \texttt{t} equal to the
first value of the sequence and stepping through each value until the
final value. The code is written so that quantities that depend on
\texttt{t} are inside the \texttt{for} loop (i.e., enclosed with in the
\texttt{\{\}} and those that do not depend on \texttt{t} are outside the
\texttt{for} loop). Note that \texttt{val} changes for different values
of \texttt{t}, \texttt{new.result} changes for different values of
\texttt{t} (because \texttt{new.result} has \texttt{time\ =\ t} and
\texttt{pop.size\ =\ val}, where \texttt{val} depends on \texttt{t}).
Finally, \texttt{df} also depends on \texttt{t}, because
\texttt{new.result} depends on \texttt{t}. In contrast, \texttt{N0} and
\texttt{lambda} do not change with \texttt{t}, so it is more efficient
to place the allocated values for these parameters outside of the loop.

We can also plot the results of our calculations:

\begin{Shaded}
\begin{Highlighting}[]
\NormalTok{lambda <-}\StringTok{ }\DecValTok{2}
\NormalTok{N0 <-}\StringTok{ }\DecValTok{10}
\NormalTok{df <-}\StringTok{ }\KeywordTok{data.frame}\NormalTok{(}\DataTypeTok{time =} \DecValTok{0}\NormalTok{, }\DataTypeTok{popn.size =} \DecValTok{10}\NormalTok{)}
\NormalTok{for(t in }\KeywordTok{seq}\NormalTok{(}\DecValTok{1}\NormalTok{,}\DecValTok{4}\NormalTok{,}\DecValTok{1}\NormalTok{))\{}
  \NormalTok{val <-}\StringTok{ }\NormalTok{N0*lambda^t}
  \NormalTok{new.result <-}\StringTok{ }\KeywordTok{data.frame}\NormalTok{(}\DataTypeTok{time =} \NormalTok{t, }\DataTypeTok{popn.size =} \NormalTok{val)}
  \NormalTok{df <-}\StringTok{ }\KeywordTok{rbind}\NormalTok{(df, new.result)}
\NormalTok{\}}
\KeywordTok{plot}\NormalTok{(df$time, df$popn.size, }\DataTypeTok{typ =} \StringTok{"l"}\NormalTok{, }\DataTypeTok{xlab =} \StringTok{"years since May 1937"}\NormalTok{, }\DataTypeTok{ylab =} \StringTok{"Population size"}\NormalTok{)}
\end{Highlighting}
\end{Shaded}

\includegraphics{BIOL-3295_files/figure-latex/unnamed-chunk-17-1.pdf}

If you already have an existing plot you can add new lines using
\texttt{lines()}. For example,

\begin{Shaded}
\begin{Highlighting}[]
\KeywordTok{plot}\NormalTok{(}\KeywordTok{seq}\NormalTok{(}\DecValTok{1}\NormalTok{,}\DecValTok{4}\NormalTok{), }\KeywordTok{c}\NormalTok{(}\DecValTok{1}\NormalTok{,}\DecValTok{3}\NormalTok{,}\DecValTok{4}\NormalTok{,}\DecValTok{2}\NormalTok{), }\DataTypeTok{ylab =} \StringTok{"y-axis"}\NormalTok{, }\DataTypeTok{xlab =} \StringTok{"x-axis"}\NormalTok{)}
\KeywordTok{lines}\NormalTok{(}\KeywordTok{seq}\NormalTok{(}\DecValTok{1}\NormalTok{,}\DecValTok{4}\NormalTok{), }\KeywordTok{seq}\NormalTok{(}\DecValTok{1}\NormalTok{,}\DecValTok{4}\NormalTok{))}
\end{Highlighting}
\end{Shaded}

\includegraphics{BIOL-3295_files/figure-latex/unnamed-chunk-18-1.pdf}

 HAND IN

\begin{itemize}
\tightlist
\item
   Write an R scipt that builds on the file you have previously make
  \emph{protection-island-q3.R}. Use the \texttt{lines()} command to add
  the predicted population size assuming geometric growth using the
  commands described in this section. If you have written the code
  correctly the result should look something like this:
\end{itemize}

\begin{Shaded}
\begin{Highlighting}[]
\NormalTok{ref.day =}\StringTok{ }\KeywordTok{c}\NormalTok{(}\DecValTok{1}\NormalTok{,}\DecValTok{1}\NormalTok{,}\DecValTok{2000}\NormalTok{)}
\CommentTok{#x = c(julian(1,1,2000, origin = ref.day), julian(1,1,2002, origin = ref.day))}
\end{Highlighting}
\end{Shaded}

\begin{enumerate}
\def\labelenumi{\arabic{enumi}.}
\setcounter{enumi}{7}
\item
  The geometric growth model is called `discrete time'. This equation is
  appropriate for populations that have regular, pulse reproduction, for
  example, pheasants that reproduce once per year in the summer. Can you
  think of examples of other species that reproduce like this?
\item
  Can you think of any species that reproduce continuously throughout
  the year? These species are better modelled with a continuous time
  model.
\end{enumerate}

\chapter{Thursday Sept 24: Doubling
times}\label{thursday-sept-24-doubling-times}

\chapter{Friday Sept 25: Density dependence and logistic
growth}\label{friday-sept-25-density-dependence-and-logistic-growth}

\section{Required reading}\label{required-reading-2}

Vandermeer, J.H., Goldberg, D.E., 2013. Population Ecology: First
Principles (Second Edition). Princeton University Press, Princeton,
United States. p9-17.
\href{https://ebookcentral-proquest-com.qe2a-proxy.mun.ca/lib/mun/detail.action?docID=1205619}{Link}

\section{Questions}\label{questions-4}

\begin{enumerate}
\def\labelenumi{\arabic{enumi}.}
\item
  What is the equation for continous time logistic growth? Define all
  the symbols in the equation.
\item
  What does dN/dt mean?
\item
  Equibrium values
\end{enumerate}

\chapter{Tuesday Sept 29: Solving the logistic growth equation using a
computer}\label{tuesday-sept-29-solving-the-logistic-growth-equation-using-a-computer}

Numerical solutions to CT logistic growth

\chapter{Thursday Oct 1: Data and the logistic
equation}\label{thursday-oct-1-data-and-the-logistic-equation}

Density dependence + data

\section{Questions}\label{questions-5}

\begin{enumerate}
\def\labelenumi{\arabic{enumi}.}
\tightlist
\item
  Question 1.9 on p12 of Vandermeer and Gordon.
\item
  Question 1.10 on p12 of Vandermeer and Gordon.
\end{enumerate}

\chapter{Fri Oct 2: Discrete time models with density dependence using a
computer}\label{fri-oct-2-discrete-time-models-with-density-dependence-using-a-computer}

\chapter{Tues Oct 6: Solving the discrete time models with density
dependence using a
computer}\label{tues-oct-6-solving-the-discrete-time-models-with-density-dependence-using-a-computer}

\chapter{Thurs Oct 8: Analysis of discrete time
models}\label{thurs-oct-8-analysis-of-discrete-time-models}

\chapter{Fri Oct 9: Density-yield and density dependence in births
versus
deaths}\label{fri-oct-9-density-yield-and-density-dependence-in-births-versus-deaths}

\chapter{Thurs Oct 15: Balsam fir}\label{thurs-oct-15-balsam-fir}

\chapter{Tues Oct 20: Stage-structured
models}\label{tues-oct-20-stage-structured-models}

\begin{itemize}
\tightlist
\item
  The idea of stage-structured models.
\item
  Multiplying matrices.
\item
  Eigenvalues of 2 x 2 matrix
\end{itemize}

\chapter{Thurs Oct 22: Stage-structured
dynamics}\label{thurs-oct-22-stage-structured-dynamics}

\begin{itemize}
\tightlist
\item
  Eigenvalues and eigenvectors
\item
  Diagrams
\end{itemize}

\chapter{Fri Oct 23: Yellow
columbine}\label{fri-oct-23-yellow-columbine}

\chapter{Tues 27-Thur 29: Midterm {[}due Fri Nov 6 at
5pm{]}}\label{tues-27-thur-29-midterm-due-fri-nov-6-at-5pm}

What is chaos? Do real populations exhibit chaos? Discuss the evidence
for and against. Provide citations.

\chapter{Fri Oct 30: Evolutionary
ecology}\label{fri-oct-30-evolutionary-ecology}

\chapter{Tues Nov 3: Evolutionary
ecology}\label{tues-nov-3-evolutionary-ecology}

\chapter{Thurs Nov 5: Evolutionary
ecology}\label{thurs-nov-5-evolutionary-ecology}

\chapter{Fri Nov 6: Evolutionary
ecology}\label{fri-nov-6-evolutionary-ecology}

\chapter{Tues Nov 10: Evolutionary
ecology}\label{tues-nov-10-evolutionary-ecology}

\chapter{Thurs Nov 12: Evolutionary
ecology}\label{thurs-nov-12-evolutionary-ecology}

\chapter{Fri Nov 13: Evolutionary
ecology}\label{fri-nov-13-evolutionary-ecology}

\chapter{Tues Nov 17: Evolutionary
ecology}\label{tues-nov-17-evolutionary-ecology}

\chapter{Thurs Nov 19: Evolutionary
ecology}\label{thurs-nov-19-evolutionary-ecology}

\chapter{Fri Nov 20: Evolutionary
ecology}\label{fri-nov-20-evolutionary-ecology}

\chapter{Population biology project
ideas}\label{population-biology-project-ideas}

\backmatter

\end{document}
